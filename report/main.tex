\documentclass[9pt, a4paper]{article}
\usepackage[margin=3.5cm]{geometry}
\usepackage{hyperref}
\usepackage{fancyhdr}
\usepackage[parfill]{parskip}

% Libertinus font family
\usepackage[oldstyle]{libertinus}
% For all the math symbols we might need.
\usepackage{unicode-math}
% For the math environments
\usepackage{amsmath}
% For moving the title up
\usepackage{titling}
\setlength{\droptitle}{-10em}

\pagestyle{fancy}
\fancyhf{}
\lhead{COM-506 - EPFL}
\rhead{Partitioning Oracle Attacks}


\date{2022-05-09}
\title{Partitioning Oracle Attacks}
\author{Leonardo \textsc{Pennino}, Louis \textsc{Bettens}}

\begin{document}

\maketitle

\section{Introduction}

When communicating over the Internet nowadays,
cryptography is almost always used to protect messages
in such an insecure environment.
It is therefore in Internet users' interest that
the cryptographic primitives used resist
known and unknown attacks as much as possible,
no matter how clever or powerful the adversary might be.
One such cryptographic primitive is AES-GCM\cite{aes-gcm},
a symmetric cipher that provides authenticated encryption.

A recent paper\cite{partitioning}
presents a practical attack against AES-GCM
that can lead to key recovery if AES-GCM is used on a public facing server.
The authors also provide insight on the theory behind their finding, 
its implications for similar cryptographic primitives,
and ways to remedy it in the short and long term.

\section{Abstract description}
\subsection{From brute force to partitioning}
To approach the concept of partitioning oracle attacks,
we will compare them to brute force attacks.
Brute force for our purposes, means
trying every possible long-term key one-by-one.
This requires a way to test if the current guess matches the real key.
We model that as an \emph{oracle}: an opaque box that tells us if it's correct, and nothing more.

Now suppose there is a way to mix $k$ keys together
to learn if the real key is amongst them or not,
for an arbitrary $k \geq 1$.
Despite being a small change in the problem,
this can make it exponentially easier to solve
using a binary search algorithm.
The oracle for this process is a \emph{partitioning} oracle since it acts on a partition of the keyspace.

\subsection{Multi-Key Collision Resistance}
Multi-key collision resistance (MKCR) is a property that AEAD ciphers can possess.
A cipher lacks resistance if a PPT adversary given $k \geq 1$ can produce a \emph{splitting} ciphertext $C$ and a set of $k$ keys $\mathcal K$ such that the ciphertext decrypts under all keys in $\mathcal K$ with non-negligible probability.

Targeted multi-key collision resistance (TMKCR) is a weaker property: the adversary is given $\mathcal K$ and must produce a ciphertext $C$ that decrypts under every key in $\mathcal K$.
A MKCR cipher is also TMKCR, but the converse is not true.

MKCR AEAD ciphers can be called \emph{robust} or \emph{key-committing}.

The method to generate multi-key ciphertexts in polynomial time
and/or in a way that is practical
depends on the specific cipher in-use.
We will evaluate the resistance of well-known AEAD ciphers
in the next section.


\section{Concrete collision}
\subsection{AES-GCM}
In the case of AES-GCM
targeted multi-key collisions can be produced using polynomial interpolation.
In fact, AES-GCM generates an authentication tag by evaluating
\begin{equation}
	T =
	C_1 \cdot H^{m+1}
	\oplus
	...
	\oplus
	C_{m-1} \cdot H^3
	\oplus
	C_m \cdot H^2
	\oplus
	L \cdot H
	\label{auth}
\end{equation}
where
$H$ is the AES encryption of the all-zero block with the shared key,
$C_{1..m}$ are the blocks of the ciphertext,
$L$ encodes the length of the ciphertext.
The symbol $\cdot$ and the exponents denote multiplication in $GF(2^{128})/(x^{128} + x^7 + x^2 +x +1)$.
This allows generating targeted multi-key collision for $k$ keys $K_{1...k}$ by solving for $C_{1...k}$ in the following system:
\begin{equation}
\begin{pmatrix}
	1 & H_1 & H_1^2 & \cdots & H_1^{k-1}
	\\
	1 & H_2 & H_2^2 & \cdots & H_2^{k-1}
	\\
	\vdots & \vdots & \vdots & \ddots & \vdots
	\\
	1 & H_k & H_k^2 & \cdots & H_k^{k-1}
\end{pmatrix}
\cdot
\begin{pmatrix}
	C_k
	\\
	C_{k-1}
	\\
	\vdots
	\\
	C_1
\end{pmatrix}
=
\begin{pmatrix}
	B_1
	\\
	B_2
	\\
	\vdots
	\\
	B_k
\end{pmatrix}
\end{equation}
where
$H_i = E_{K_i}(0^{128})$
,
$B_i = (L \cdot H_i \oplus E_{K_i}(IV \parallel 0^{31} \parallel 1) \oplus T) \cdot H_i^{-2}$.
,
IV is a 96-bit nonce
and
$T$ is arbitrary.
Since the matrix is a Vandermonde matrix, it can be inverted in $O(k^2)$, which makes the algorithm more efficient.
In this way, $IV$, $T$ and $C_{1...k}$ form a valid ciphertext for which equation \ref{auth} will hold under any key in $K_{1...k}$, and for any other key only with negligible probability.
This fits the definition of a targeted multi-key collision.

We note that the length of the splitting ciphertext is proportional to $k$.
This is important because
if the key is distributed uniformly in $\{0,1\}^{128}$,
we need to send about $2^{128}$ AES blocks through the oracle.
Thus, this attack is not practical in that case.
If the key is derived from a human-selected password however,
attacks are feasible as we will see.

\subsection{AES-GCM-SIV}
AES-GCM-SIV is a cipher similar to AES-GCM that is designed to mitigate nonce-reuse.
It is not designed to be a key-committing AEAD cipher.
Despite that, the authors have found \emph{non-targeted} multi-key collision attacks on AES-GCM-SIV
using polynomial interpolation. With AES-GCM, those could be targeted collisions.

\subsection{Poly1305}
AEAD ciphers such as Chacha20-Poly1305 use a similar algebraic construct to generate MACs.
The algorithm however operates in $GF(2^{130}-5)$, and has some additional subtleties that make it much harder to implement multi-key collision attacks.
The authors were unable to produce splitting ciphertexts valid under more than 10 different keys. However, this does not prove that this family of cipher are key-committing.

\section{Real-world attack}
Shadowsocks\cite{shadowsocks} is a TCP and UDP proxy protocol that can use AES-GCM.
It can be configured to use keys derived from a password.
Each packet starts with a random salt.
That salt is passed along with the password to a HKDF,
and the output is used as a symmetric cipher key.

The authors discovered that it was possible to send a splitting ciphertext as a
UDP packet to be forwarded and observe whether the server accepted it or not.
The attacker can send the splitting ciphertext as a UDP packet to be relayed,
and spoof a response on the server's ephemeral UDP ports.
If ciphertext is valid, the server expected a response and will relay it to the attacker.
This creates a partitioning oracle.

A splitting ciphertext can only target up to 4091 passwords at once because the size of UDP packets is limited.
Despite that,
the authors showed in a simulation that,
compared to a 1-by-1 online password guessing attack,
the partitioning oracle attack had a significantly higher success rate and consumed less bandwidth despite the size of the packets.

After being informed of the problem,
Shadowsocks contributors disabled UDP functionality by default.

\section{Possible mitigations}
The authors recommend that robust AEAD ciphers be standardized and made available to application developers so that these kinds of attacks do not occur. The algorithms should be efficient in order to maximize adoption. according to them, key-committing AEAD should be the default.

In the shorter term, applications that combine passwords and AEAD may need to adopt mitigations.
One possible mitigation is
limiting the size of acceptable ciphertexts.
Of course this is not possible for every application,
but it is the approach chosen by the age file encryption tool.
\footnote{\url{https://github.com/FiloSottile/age/commit/2194f69}}

Another available mitigation is
to use synchronous password-authenticated key exchange protocols.
This is only possible in interactive contexts
and requires a PAKE protocol that resists partitioning oracle attacks.
The authors examine well-known PAKE protocols and show that some are safe and some are vulnerable.
Shadowsocks could implement this mitigation at the cost of backwards-compatibility.

\bibliographystyle{IEEEtran}
\footnotesize \bibliography{bib}
\end{document}
